\section{Inserzione di idrogeno in \ce{WO3} e  test di conducibilità}
\subsection{Sintesi}
\subsubsection{Procedura sperimentale}

Abbiamo unito 50 mL di acido cloridrico 3 M ad un becher contenente 0.5 g di triossido di tungsteno e quindi abbiamo aggiunto 1 g di limatura di zinco. Al termine dell’effervescenza abbiamo filtrato su Buchner e raccolto il contenuto in una provetta. Abbiamo ottenuto 0.5123 g di prodotto, ma non essendo il prodotto di stechiometria nota non è possibile calcolare la resa della reazione.




\subsubsection{Commenti e osservazioni}


Il solido prodotto si presentava come sostanza nero antracite con sfumature blu. Il colore che abbiamo ottenuto è sintomo di una reazione ben riuscita; si veda \cite{conduzione}. Polverosa con particelle molto fini.
\ce{WO3} era molto denso così come la polvere ottenuta. %Peroskiti hanno proprietà di conduttori e semiconduttori.

\subsection{Test conducibilità}

Per caratterizzare il composto sintetizzato abbiamo eseguito due prove di conducibilità elettrica. I dati raccolti sono stati raccolti nella \autoref{tab:conduci} nella \autoref{sec:dati}.





